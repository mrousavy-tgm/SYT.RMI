%!TEX root=../document.tex

\section{Ergebnisse}
\label{sec:Ergebnisse}

Der Grundaufbau besteht aus einem Client Java Programm, welches über das Remote Method Invocation ("RMI") Protokoll einen Server kontaktiert, welcher dann wiederum eine beliebige Aufgabe für diesen Client ausführt.

Diese Aufgabe ("Task") soll generisch sein, sodass verschiedene Implementationen von beliebigen Aufgaben ausgeführt werden können, wie beispielsweise die Berechnung von Pi oder der Fibonacci Nummer.

Um die Aufgabe zu erweitern, wird ein Proxy zwischen die Clients und Server geschalten, welcher dann einzelne Anfragen der Clients mittels Round Robin Konzept gleichgerecht auf die einzelnen Server verteilt.

\subsection{Projekt}
Als Entwicklungsumgebung wird IntelliJ IDEA von JetBrains [1] verwendet. Es wird ein neues Projekt erstellt, mit folgenden packages/submodules:
\begin{itemize}
    \item \textbf{Client}: Client-side RMI implementierung
    \item \textbf{Server}: Server-side RMI implementierung, inkl. Load Balancer
    \item \textbf{Modules}: Shared-code, u.a. die generische \textbf{Task<T>} klasse, Pi und Fibonacci implementierungen, etc.
\end{itemize}


\subsection{Module}
Bevor der Client oder Server implementiert werden kann, muss eine basis gebaut werden. Es werden Funktionen implementiert, welche Client \textbf{und} Server verwenden. Unter anderem gehört dazu:

\subsubsection{Compute}
Es muss ein Interface definiert werden, welches als Stub [1] dient, also die Schnittstelle zur Implementierung für den \textbf{Task} auf einer anderen Codebase.

Die Implementierung ist sehr simpel, es besteht aus einem Interface mit der \textbf{run()} Funktion:

\begin{lstlisting}[style=Java, caption=Module Implementation - Compute interface]
public interface Compute extends Remote, Serializable {
    <T> T run(Task<T> t) throws RemoteException;
}
\end{lstlisting}

\subsubsection{Task}
Es muss ein Generisches Interface definiert werden, welches auch nur eine \textbf{run()} Funktion aufweist:

\begin{lstlisting}[style=Java, caption=Module Implementation - Task interface]
public interface Task<T> {
	T run();
}
\end{lstlisting}

Dieses Interface ist dazu da, um jeden beliebigen Auftrag auszuführen, ohne die Implementierung (bzw. \textbf{run()}-Methode) zu kennen.

\subsubsection{Fibonacci}
Nun kann eine Tatsächliche Aufgabe implementiert werden, als erstes Beispiel wird Fibonacci verwendet.

Die Klasse Fibonacci soll aus \textbf{Task<BigInteger>} erben da es ein Task mit dem Rückgabewert BigInteger ist, sowie aus \textbf{Serializable}, damit aus dem Memory void* ein serialisiertes byte array gebaut werden kann.

Das serialisierte byte array kann somit über die Netzwerkschnittstelle gesendet werden.

Für die Serilization wird außerdem eine statische, final Konstante aus einem 64bit Integer erstellt, welche quasi einen Identifier darstellt welcher der Serilisation erkennbar gibt, dass es sich um die selbe Implementierung einer Klasse handelt.

Die \textbf{run()} Methode schaut folgendermaßen aus:

\begin{lstlisting}[style=Java, caption=Module Implementation - Fibonacci run]
@Override
public BigInteger run() {
	BigInteger previous = BigInteger.ONE;
    BigInteger recent = BigInteger.ONE;
    for (int i = 0; i < _digits; i++) {
    	BigInteger temp = previous.add(recent);
        previous = recent;
        recent = temp;
	}
	return recent;
}
\end{lstlisting}


\subsection{Client}


Es kann gut möglich sein, dass Lehrkräfte hier auch noch andere Eckpunkte explizit verlangen. Diese können dann in der selben Hierarchiestufe wie die \textit{\nameref{sec:Ergebnisse}} eingeordnet werden. Viel Spass nun mit einer kleinen Übersicht von \LaTeX-Elementen.

\subsection{Tabelle}
\renewcommand{\arraystretch}{1.5}
\begin{table}[!h]
	\center
	\begin{tabular}{ | @{\hspace{3mm}} c @{\hspace{3mm}} | @{\hspace{3mm}} l @{\hspace{3mm}} | }
		\hline Header & Kopf\\ \hline\hline
		\textbf{Lorem} & Ipsum dolor sit amet, consetetur sadipscing elitr\\ \hline
		\textbf{Ipsum} & At vero eos et accusam et justo duo dolores et ea rebum.\\
			& Stet clita kasd gubergren, no sea takimata sanctus\\ \hline
		\textbf{Dolor} & Consetetur sadipscing elitr, sed diam nonumy\\\hline
	\end{tabular}
	\caption{Lorem ipsum dolor sit amet \cite{tanenbaum2007verteilte}}
	\label{methoden}
\end{table}

\subsection{Aufzählung}

\begin{itemize}
	\item \textbf{Lorem ipsum:} dolor sit amet, consetetur sadipscing elitr
	\item sed diam nonumy eirmod tempor invidunt ut labore et dolore magna aliquyam erat
	\item ut labore et dolore magna aliquyam erat, sed diam voluptua
\end{itemize}

\subsection{Code}

At vero eos et accusam et justo duo dolores et ea rebum.

\begin{lstlisting}[style=Java, caption=Implizite Transaktion \cite{tanenbaum2007verteilte}]
try{
   gTransCur.begin();
   //Perform the operation inside the transaction
   not_registered = 
       gRegistrarObjRef.register_for_courses(student_id,selected_course_numbers);


   if (not_registered != null)

     //If operation executes with no errors, commit the transaction
     boolean report_heuristics = true;
     gTransCur.commit(report_heuristics);

   } else gTransCur.rollback();


} catch(org.omg.CosTransactions.NoTransaction nte) {
    System.err.println("NoTransaction: " + nte);
    System.exit(1);
} catch(org.omg.CosTransactions.SubtransactionsUnavailable e) {
    System.err.println("Subtransactions Unavailable: " + e);
    System.exit(1);
} catch(org.omg.CosTransactions.HeuristicHazard e) {
    System.err.println("HeuristicHazard: " + e);
    System.exit(1);
} catch(org.omg.CosTransactions.HeuristicMixed e) {
    System.err.println("HeuristicMixed: " + e);
    System.exit(1);
}
\end{lstlisting}

